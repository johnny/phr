\documentclass{article}
\usepackage[utf8]{inputenc}
\begin{document}
{\bf Aufgabe 17}\\
{\bf a)}\\
Die reine Kommunikationszeit in Abhängikeit der Problemgröße N errechnet sich nach folgendem Schema\ldots \\
\ldots für einen Schritt ($t_k$ ist die Übertragungszeit):
$$t_i(n) = \frac{n}{2^i}*t_k$$
\ldots für die Gesamtheit aller Schritte:
$$t(N,P) = N + \sum_{i=0}^{\frac{P}{2}-1}{\frac{n}{2^i}*t_k}$$
\\
{\bf b)}\\
Die Rechenzeit in Abhängigkeit der Problemgröße N errechnet sich nach folgendem Schema\ldots \\
\ldots für einen Schritt ($t_r$ ist Zeit für eine Multiplikation):
$$t(N) = (N\log{N})*t_r$$
\ldots für sie Gesamtheit aller Rechenschritte:
$$t_{ges}(N,P) = N + \sum_{i=0}^{\frac{P}{2}-1}{\frac{N}{2^i}}t_k + (\frac{N\log{N}}{P})t_r$$
\\
{\bf c)}\\
Für den SpeedUp ergibt sich:
$$S(N,P) = \frac{N\log{N}}{N + \sum_{i=0}^{\frac{P}{2}-1}{\frac{N}{2^i}}t_k + (\frac{N\log{N}}{P})t_r}$$
{\bf d)}
Daraus resultiert folgende Isoeffizienzfunktion:\\
(?)
\end{document}
