\documentclass{article}
\usepackage[utf8]{inputenc}
\begin{document}
\pagestyle{myheadings}
\markright{PHLR - Blatt 8, Jonas von Andrian, Thomas Monninger}
{\bf Aufgabe 17}\\
{\bf a)}\\
Die reine Kommunikationszeit in Abhängikeit der Problemgröße N errechnet sich nach folgendem Schema\ldots \\
\ldots für einen Schritt ($t_k$ ist die Übertragungszeit):
$$t_i(n) = \frac{n}{2^i}*t_k$$
\ldots für die Gesamtheit aller Schritte:
$$t(N,P) = N + \sum_{i=0}^{\frac{P}{2}-1}{\frac{n}{2^i}*t_k}$$
\\
{\bf b)}\\
Die Rechenzeit in Abhängigkeit der Problemgröße N errechnet sich nach folgendem Schema\ldots \\
\ldots für einen Schritt ($t_r$ ist Zeit für eine Multiplikation):
$$t(N) = (N\log{N})*t_r$$
\ldots für sie Gesamtheit aller Rechenschritte:
$$t_{ges}(N,P) = N + \sum_{i=0}^{\frac{P}{2}-1}{\frac{N}{2^i}}t_k + (\frac{N\log{N}}{P})t_r$$
\\
{\bf c)}\\
Für den SpeedUp ergibt sich:
$$S(N,P) = \frac{N\log{N}}{N + \sum_{i=0}^{\frac{P}{2}-1}{\frac{N}{2^i}}t_k + (\frac{N\log{N}}{P})t_r}$$
{\bf d)}
Exisitiert eine Isoeffizienzfunktion muss gelten:
$$E(W,P) =  \frac{1}{1+\frac{T_o(W,P)}{W}}$$
und
$$\lim\limits_{W \rightarrow \infty} E(W,P)=1$$
und damit muss gelten, dass bei festem P der Overhead weniger als linear 
anwächst. D.h., es muss gelten
$$T_o(W,P)|_P=const < O(W)$$
Das ist beim parallelen Mergesort der Fall, da für jeden zusätzlichen 
Knoten (bzw. jedes Paar, es kommen ja immer 2 Knoten pro Schritt hinzu) 
der Overhead abnimmt.\\
Die gewünschten Testergebnisse traten leider nicht ein, da auf den 
Poolrechnern das Programm immer mit einem Fehler abbricht (MPI\_RECV : 
Message truncated). 
Nachforschungen haben ergeben, dass es wohl an einem 
Bug in der MPI-Implementierung liegt, da es auf dem lokalen Rechner 
läuft (jedoch trat auch hier nicht das gewünschte Laufzeitverhalten 
ein).
\end{document}
