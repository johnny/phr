\documentclass{article}
\usepackage{tabularx}
\usepackage[utf8]{inputenc}
\begin{document}
\pagestyle{myheadings}
\markright{PHLR - Blatt 9, Jonas von Andrian, Thomas Monninger}
{\bf Aufgabe 19}\\
{\bf a)}\\
\begin{tabularx}{\textwidth}{|X|X|X|X|}
\hline
 & Ring & Feld & Hypercube \\
\hline
One-to-all & $(t_s + t_w \cdot n) ld P + t_h(P-1)$ & $ld P(t_s + t_w \cdot n) + t_h \cdot 2(\sqrt{P}-1)$ & $(t_s+t_h+t_w \cdot n) ld P$ \\
\hline
All-to-all-broadcast & $2 ld P(t_s+t_h)+2t_wn(P-1)$ & $2 ld P(t_s+t_h)+2t_wn(P-1)$ & $2 ld P(t_s+t_h)+2t_wn(P-1)$ \\
\hline
One-to-all mit individuellen Nachrichten & & & $t_wn(P-1)$ \\
\hline
All-to-all mit individuellen Nachrichten & & & $2(t_s+t_h) ld P+t_wnP ld P$ \\
\hline
\end{tabularx}
{\bf b)}\\
Vorteile des \textit{Cut-Through-Routing} gegenüber dem \textit{Store-And-Forward-Routing} ergeben sich für die Ring- und 2D-Feld-Topologie, jedoch nicht für den Hypercube.

{\bf Aufgabe 21}\\
{\bf a)}\\
$t_{seq}=O(4n)$\\
{\bf b)}\\
$n_{seq}=13$\\
$n_{par}=4$\\
{\bf c)}\\
$n_{seq}=6$\\
$n_{par}=6$\\
{\bf d)}\\
$S_b = \frac{13}{5}$\\
$S_c = 1$\\
Ja, es kann hierbei Anomalien im Speedup geben. Auf der linken Seite der Abbildung 03 ist, wie man im Speedup $S_b$ sieht, bereits eine Anomalie des Speedups zu erkennen.\\
{\bf e)}\\
Ein superlinearer Speedup tritt dann auf, wenn der linke Teilbaum groß ist und sich der gesuchte Knoten im rechten Teilbaum befindet. Das parallele Verfahren bricht ab, sobald der gesuchte Knoten im rechten Teilbaum gefunden 
wurde. Das sequenzielle Verfahren hingegen läuft (nahezu) endlos weiter.
\end{document}
